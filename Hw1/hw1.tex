\PassOptionsToPackage{quiet}{xeCJK}
\documentclass[12pt]{article}
\usepackage{xeCJK}
\title{Hw1}
\author{109062135陳家輝}
\date{\today}
\begin{document}

\maketitle
\noindent
\textbf{
    1. If the diameter of a network with 100 nodes is 1, what is the minimum number of links in this network?
}
\begin{flushleft}
    Since the diameter of network is 1, that is,
    any two of them must hop to each other directly.
    Therefore, any two of them must have a link between them.
    So, the minimum number of links in this network is:
\end{flushleft}

\begin{center}
    $C_{2}^{100}=\frac{100*99}{2}=4950 $
\end{center}


\noindent
\textbf{
    2. If the diameter of a network with 100 nodes is 2, what is the minimum number of links in this network?
}
\begin{flushleft}
    \qquad
    Since the diameter of network is 2,
    any two of them, except by the mid node, can hop to each other by passing the mid node.
    And, it only takes 99 links to link the mid node and the others node.

    \qquad
    So, the minimum number of links in this network is 99.
\end{flushleft}

\noindent
\textbf{
    3. For a network of 100 nodes, if the degree of every node is at most 2, what is the minimum diameter of that network?
}

\begin{flushleft}
    \qquad
    When the degree of every node is 2, that is, any of them can link other 2 nodes.
    Therefore, we can construct the network into a cycle.
    Then, when a node wants to hop to any other node, it can go left or go right.
    But whatever it goes, it takes at most 50 steps to get there.
    For example, if the number 1 node wants to hop to the number 25 node,
    it can hop to 2, then 3...until it gets to 25.
    If it wants to hop to the number 75 node,
    it can hop to 100, then 99...until it gets to 75.

    \qquad
    So, the minimum diameter of this network is 50.
\end{flushleft}

\noindent
\textbf{
    4. For a network of 100 nodes, if the degree of every node is at most 3, is it possible that the diameter of this network is not greater than 5?
}

\begin{flushleft}
    \qquad
    Consider a binary tree but level 2 has three nodes, that is, root has three children.
    Then from level 2, every nodes has two children.
    This tree consist of 6 level, total nodes in this tree is:
    \begin{center}
        $ 1 + 3*2^0 + 3*2^1 + 3*2^2 + 3*2^3 + 3*2^4 = 94 $
    \end{center}
    \qquad
    Clearly, this tree can not contain all 100 nodes, thus,
    there exist some nodes that if we want to hop from one node to another,
    we will go exceed 5 steps.
    So, it is impossible that the network with degree of every node is 3 has diameter not greater than 5.
\end{flushleft}

\end{document}